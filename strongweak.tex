\documentclass[10pt]{amsproc}
\usepackage[a4paper,hmargin={4cm,4cm},vmargin={3.5cm,3.5cm}]{geometry}
\usepackage[colorlinks=true,bookmarksopen=false,linkcolor=blue,citecolor=red]{hyperref}
\usepackage{calc} 
\usepackage{datetime}
\usepackage{comment}
\usepackage{amssymb}
\usepackage{amsthm}
\usepackage{amsmath}
\usepackage{amsrefs}
\usepackage{tcolorbox}
\usepackage{dsfont}
\usepackage{euscript}
\usepackage{fourier-orns}
\usepackage{datetime2}
\usepackage{palatino}
\usepackage[sc]{mathpazo}
\usepackage[T1]{fontenc}
\usepackage{graphicx}

\usepackage{tikz-cd}
\tikzcdset{
arrow style=tikz,
diagrams={>=latex}
}

\linespread{1.1}
\setlength{\parindent}{0ex}
\setlength{\parskip}{.4\baselineskip}
\definecolor{brown}{RGB}{150, 50, 10}
\definecolor{green}{RGB}{5,110, 35}

\DeclareFontFamily{OT1}{pzc}{}
\DeclareFontShape{OT1}{pzc}{m}{it}{<-> s * [1.10] pzcmi7t}{}
\DeclareMathAlphabet{\mathpzc}{OT1}{pzc}{m}{it}

\newcommand{\mylabel}[1]{{\ssf{#1}}\hfill}
\renewenvironment{itemize}
  {\begin{list}{}{%
   \setlength{\parskip}{0mm}
   \setlength{\topsep}{.2\baselineskip}
   \setlength{\rightmargin}{0mm}
   \setlength{\listparindent}{0mm}
   \setlength{\itemindent}{0mm}
   \setlength{\labelwidth}{2ex}
   \setlength{\itemsep}{.1\baselineskip}
   \setlength{\parsep}{0mm}
   \setlength{\partopsep}{0mm}
   \setlength{\labelsep}{1ex}
   \setlength{\leftmargin}{\labelwidth+\labelsep}
   \let\makelabel\mylabel
   }}
   {\vspace*{-.3\baselineskip}\end{list}}

\def\E{\exists}
\def\A{\forall}
\def\mdot{\mathord\cdot}
\def\models{\vDash}
\def\notmodels{\nvDash}
\def\proves{\vdash}
\def\notproves{\nvdash}
\def\ZZ{\mathds Z}
\def\NN{\mathds N}
\def\QQ{\mathds Q}
\def\RR{\mathds R}
\def\BB{\mathds B}
\def\CC{\mathds C}
\def\PP{\mathds P}
\def\Ar{{\rm Ar}}
\def\dom{\textrm{\rm dom}}
\def\range{\textrm{\rm rng}}
\def\rank{\textrm{\rm rank}}
\def\dcl{\textrm{\rm dcl}}
\def\acl{\textrm{\rm acl}}
\def\bdd{\textrm{\rm bdd}}
\def\fp{{\rm fp}}
\def\cf{\textrm{\rm cf}}
\def\rad{\textrm{\rm rad}}
\def\eq{{{\rm eq}}}
\def\ccl{{\rm ccl}}
\def\Th{{\rm Th}}
\def\Diag{{\rm Diag}}
\def\Mod{\textrm{\rm Mod}}
\def\Rmod{{\mbox{\scriptsize $R$-mod}}}
\def\Aut{{\rm Aut\kern.15ex}}
\def\Autf{\mathord{\rm Aut\kern.15ex{f}\kern.15ex}}
\def\Cb{{\rm Cb\kern.15ex}}
\def\orbit{\O}
\def\oorbit{\mathpzc{o}}
\def\oorbitf{\mathpzc{of\!}}
\def\id{{\rm id}}
\def\tp{{\rm tp}}
\def\qftp{{\rm qf\mbox{-}tp}}
\def\attp{{\rm at\mbox{-}tp}}
\def\atpmtp{\mbox{\rm at$^{\scriptscriptstyle\pm}$-tp}}
\def\Deltatp{\Delta\mbox{\rm -tp}}
\def\pmDelta{\smash{\Delta\hskip-.4ex\raisebox{.8ex}{$\scriptscriptstyle\pm$}}}
\def\BDelta{\smash{\Delta\hskip-.4ex\raisebox{.8ex}{\tiny\sf B}}}
\def\GDelta{\smash{\Delta\hskip-.4ex\raisebox{.8ex}{\tiny\sf G}}}
\def\LDelta{L\kern-.5ex^{\scriptscriptstyle\Delta}}
\def\UDelta{\U\kern-.2ex^{\scriptscriptstyle\Delta}}
\def\pmDeltatp{\noindent\pmDelta\hskip-.3ex{\rm -tp}}
\def\EMtp{\mbox{{\small EM}-tp}}

\def\nonfork{\mathop{\raise0.2ex\hbox{\ooalign{\hidewidth$\vert$\hidewidth\cr\raise-0.9ex\hbox{$\smile$}}}}}

\def\cnonfork{\mathbin{\raise1.8ex\rlap{\kern0.6ex\rule{0.6ex}{0.1ex}}\rlap{\kern1.1ex\rule{0.1ex}{1.9ex}}\raise-0.3ex\hbox{$\smile$} } }

\def\nonforkc{\mathbin{\raise1.8ex\rlap{\kern1.1ex\rule{0.6ex}{0.1ex}}\rlap{\kern1.1ex\rule{0.1ex}{1.9ex}}\raise-0.3ex\hbox{$\smile$} } }

\def\QED{}
\def\cpaw{\mathbin{\ooalign{\kern-0.4ex$-$\hidewidth\cr$<$}}}
\def\cpawdot{\ooalign{$\kern1.2ex\cdot$\cr$\cpaw$\cr}}
\def\cev#1{\reflectbox{\ensuremath{\vec{\reflectbox{\ensuremath{#1}}}}}}

\def\sm{\smallsetminus}
\def\atpmL{L_{\rm at^{\scriptscriptstyle\pm}}}
\def\qfL{L_{\rm qf}}
\def\atL{L_{\rm at}}
\def\simdiff{\triangle}
\def\IMP{\Rightarrow}
\def\PMI{\Leftarrow}
\def\IFF{\Leftrightarrow}
\def\NIFF{\nLeftrightarrow}
\def\imp{\rightarrow}
\def\pmi{\leftarrow}
\def\iff{\leftrightarrow}
\def\niff{\mathrel{{\leftrightarrow}\llap{\raisebox{-.1ex}{{\small$/$}}\hskip.5ex}}}
\def\nequiv{\mathrel{\mbox{$\equiv$\llap{{\small$/$}\hskip.3ex}}}}
\def\equivEM{\stackrel{\smash{\scalebox{.5}{\rm EM}}}{\equiv}}
\def\equivL{\stackrel{\smash{\scalebox{.5}{\rm L}}}{\equiv}}
\def\equivKP{\stackrel{\smash{\scalebox{.5}{\rm KP}}}{\equiv}}
\def\equivSh{\stackrel{\smash{\scalebox{.5}{\rm Sh}}}{\equiv}}

\def\isomap{\mathrel{\rlap{\kern0.7ex\raisebox{.7ex}{\scriptsize$\kern.2ex\sim$}}\rightarrow}}

\def\P{\EuScript P}
\def\D{\EuScript D}
\def\Aa{\EuScript A}
\def\Ee{\EuScript E}
\def\X{\EuScript X}
\def\Y{\EuScript Y}
\def\Z{\EuScript Z}
\def\C{\EuScript C}
\def\U{\EuScript U}
\def\Hh{\EuScript H}
\def\I{\EuScript I}
\def\V{\EuScript V}
\def\W{\EuScript W}
\def\R{\EuScript R}
\def\F{\EuScript F}
\def\G{\EuScript G}
\def\B{\EuScript B}
\def\M{\EuScript M}
\def\Ll{\EuScript L}
\def\K{\EuScript K}
\def\O{\EuScript O}
\def\J{\EuScript J}
\def\S{\EuScript S}
\def\<{\langle}
\def\>{\rangle}
\def\0{\varnothing}
\def\theta{\vartheta}
\def\phi{\varphi}
\def\epsilon{\varepsilon}
\def\ssf#1{\textsf{\small #1}}

\newtheoremstyle{mio}% name
     {2\parskip}     % Space above
     {2\parskip}     % Space below
     {}%         Body font
     {}%         Indent amount (empty = no indent, \parindent = para indent)
     {\bfseries}% Thm head font
     {}%        Punctuation after thm head
     {1ex}%     Space after thm head: " " = normal interword space;
           %   \newline = linebreak
     {\llap{\thmnumber{#2}\hskip0.9ex}\thmname{#1}\thmnote{\bfseries{}#3}}% Thm head spec (can be left empty, meaning `normal')

\newcounter{thm}

\renewcommand{\thethm}{\arabic{thm}}

\tcbset{
  mythm/.style={
    colback=black!5!white,
    colframe=white!50!black,
    extrude right by=1ex,
    extrude left by=6.5ex,
    % before=\par\noindent,
    % after=\par\noindent,
    top=0.7ex,
    bottom=1.5ex,
    right=1ex,
    left=6.5ex,
    boxsep=0pt,
    parbox=false,
    before skip=\baselineskip,
    after skip=\baselineskip,
  },
}
\theoremstyle{mio}
\newtheorem{theorem}[thm]{Theorem}\tcolorboxenvironment{theorem}{mythm}
\newtheorem{corollary}[thm]{Corollary}\tcolorboxenvironment{corollary}{mythm}
\newtheorem{proposition}[thm]{Proposition}\tcolorboxenvironment{proposition}{mythm}
\newtheorem{lemma}[thm]{Lemma}\tcolorboxenvironment{lemma}{mythm}
\newtheorem{fact}[thm]{Fact}\tcolorboxenvironment{fact}{mythm}
\newtheorem{conjecture}[thm]{Conjecture}\tcolorboxenvironment{conjecture}{mythm}
\newtheorem{definition}[thm]{Definition}\tcolorboxenvironment{definition}{mythm}
\newtheorem{assumption}[thm]{Assumption}\tcolorboxenvironment{assumption}{mythm}
\newtheorem{void}[thm]{}\tcolorboxenvironment{void}{mythm}
\newtheorem{remark}[thm]{Remark}\tcolorboxenvironment{remark}{mythm}
\newtheorem{notation}[thm]{Notation}\tcolorboxenvironment{notation}{mythm}
\newtheorem{note}[thm]{Note}\tcolorboxenvironment{note}{mythm}
\newtheorem{exercise}[thm]{Exercise}
\newtheorem{question}[thm]{Question}
\newtheorem{example}[thm]{Example}

\makeatletter
\providecommand{\proofNameStyle}{\bfseries}
\renewenvironment{proof}[1][\proofname]{\par
  \pushQED{\qed}%
  \normalfont% \topsep6\p@\@plus6\p@\relax
  % \vspace*{-\baselineskip}
  \trivlist
  \item[\hskip\labelsep
        \proofNameStyle
    #1\@addpunct{.}]\ignorespaces
}{%
  \popQED\endtrivlist\@endpefalse
}
\makeatother

% \pagestyle{plain}

\definecolor{green}{RGB}{5,110, 35}
\definecolor{emphcolor}{rgb}{.90,.98,.98}

\def\mr{\color{brown}}
\def\gr{\color{green}}
\def\vl{\color{violet}}

\def\mrA{{\mr\Aa}}
\def\mrB{{\mr\B}}
\def\mrC{{\mr\C}}
\def\mrD{{\mr\D}}
\def\mrG{{\mr\G}}
\def\mrU{{\mr\U}}
\def\mrV{{\mr\V}}
\def\mrW{{\mr\W}}
\def\mrX{{\mr\X}}
\def\grA{{\gr\Aa}}
\def\grB{{\gr\B}}
\def\grC{{\gr\C}}
\def\grD{{\gr\D}}
\def\grG{{\gr\G}}
\def\grZ{{\gr\Z}}
\def\grW{{\gr\W}}
\def\grV{\V^{\gr z}}

\renewcommand*{\emph}[1]{%
\kern-0.3ex
\smash{\tikz[baseline]\node[rectangle, fill=black!20!yellow!50!white, rounded corners, inner xsep=0.5ex, inner ysep=0.2ex, anchor=base, minimum height = 2.7ex]{#1};}\kern-0.5ex
}

%%%%%%% GETCOMMIT
\newcommand\dotGitHEAD{}
\newcommand\branch{}
\newcommand\commit{}
\makeatletter
\let\myfilehandle\@inputcheck\makeatother

\openin\myfilehandle=.git/HEAD\relax

\begingroup\endlinechar-1
  \global\read\myfilehandle to \dotGitHEAD
\endgroup
\closein\myfilehandle

\newcommand\GetBranch{}
\def\GetBranch ref: refs/heads/#1\relax{\renewcommand{\branch}{#1}}

\expandafter\GetBranch\dotGitHEAD\relax
\makeatother
%%%%%%% END GETCOMMIT

% \author{S. A. Polymath}

\begin{document}
% \author{G.\ A.\ M.\ Polymath}
\title{Group actions on models}
\hfill\texttt{Branch:\ \branch\ \DTMnow}
\raggedbottom
\begin{abstract}
  \setlength{\parindent}{0ex}
  \setlength{\parskip}{.4\baselineskip}
  A set is \textit{strongly generic\/} if the intersection of any finitely many of its translations is generic.
  %To demonstrate the convenience of this notion I use it for a short proof of (a generalization of) Newelski's theorem on the diameter of the Lascar graph, see Theorem~\ref{thm_newelski}.

  Theorem~\ref{thm_coalesce} shows that the condition \textit{strongly generic = generic\/} is robust.
  It might be of some interest (it is reminiscent of \textit{forking = dividing\/}).
  Does it have interesting examples?

  Proposition~\ref{prop_stabilizer1} looks interesting. 
  Can it be strengthened to obtain a type-definable group?
  
  The connections with topological dynamics have not been explored :(
\end{abstract}
\maketitle

\def\medrel#1{\parbox[t]{5ex}{$\displaystyle\hfil #1$}}
\def\ceq#1#2#3{\parbox[t]{17ex}{$\displaystyle #1$}\medrel{#2}{$\displaystyle #3$}}

%%%%%%%%%%%%%%%%%%%%%%%%%%
%%%%%%%%%%%%%%%%%%%%%%%%%%
%%%%%%%%%%%%%%%%%%%%%%%%%%
%%%%%%%%%%%%%%%%%%%%%%%%%%
%%%%%%%%%%%%%%%%%%%%%%%%%%

\section{The dual perspective on invariance}\label{dual_perspective}

The notions in this section are well-known but sometimes the terminology differs.

In this chapter $\Delta\subseteq L_{{\mr x}\,{\gr z}}(\U)$.
Let $\grZ\subseteq\U^{\gr z}$.
We write \emph{$\Delta(\grZ)$} for the set of formulas of the form $\phi({\mr x}\,;{\gr b})$ for some $\phi({\mr x}\,;{\gr z})\in\Delta$ and some ${\gr b}\in\grZ$.
We write \emph{$\pmDelta(\grZ)$} for the set of formulas in $\Delta(\grZ)$ or negation thereof.
Furthermore, we write $S_\Delta(\grZ)$ for the set of complete $\pmDelta(\grZ)$-types.

We write \emph{$\BDelta(\grZ)$} for the set of Boolean combinations of formulas in $\Delta(\grZ)$.

% When $\grZ=\Aa^{\gr z}$ for some $\Aa\subseteq\U$, we may write $\Aa$ for $\Aa^{\gr z}$ in the notation above.

% Finally, define \emph{$\GDelta(A)$\/} to be the set of formulas $\phi({\mr x})\in L(\U)$ that are equivalent to some formula in $\BDelta(A)$ or, equivalently, that are invariant over $A$.
% In the literature these formulas are called \textit{generalized\/} $\Delta$-formulas over $A$.
% Note that when $A$ is a model $\GDelta(A)$-formulas are equivalent to $\BDelta(A)$-formulas.

\begin{assumption}\label{notation_GXphi}
  Let $G$ be a group that acts on some sets ${\mrX}\subseteq\U^{\mr x}$ and ${\grZ}\subseteq\U^{\gr z}$.
  We require that for every $\phi({\mr x}\,;{\gr z})\in\Delta$ the set $\phi(\mrX;\grZ)$ is invariant under the action of $G$.
  For convenience, we will assume that $G$ is the identity outside ${\mrX}$ and ${\grZ}$.

  When $p({\mr x})\subseteq\BDelta(\grZ)$ and  ${\mrD}\subseteq\U^{\mr x}$ we write $p({\mr x})\proves{\mr x}\in{\mrD}$ if the inclusion $\psi(\mrX)\subseteq{\mrD}$ for some $\psi({\mr x})$ that is conjunctions of formulas in $p({\mr x})$.
\end{assumption}

Let $\grD\subseteq\U^{\gr z}$.
We say that $\grD$ is \emph{invariant\/} under the action of $G$, or \emph{$G$-invariant,} if  $\grD$ is fixed setwise by $G$.
That is, $g\,\grD=\grD$ for every $g\in G$.
Yet in other words, if

\ceq{\ssf{is1.}\hfill {\gr a}\in\grD}{\iff}{g{\gr a}\in\grD}\hfill for every ${\gr a}\in\grZ$ and every $g\in G$.

A formula is $G$-invariant if the set it defines is $G$-invariant.
We say that $p({\mr x})\subseteq\BDelta(\grZ)$ is \emph{invariant\/} under the action of $G$, or \emph{$G$-invariant,} if for every $\BDelta(\grZ)$-formula $\theta({\mr x}\,;{\gr\bar a})$. 

\ceq{\ssf{it1.}\hfill\theta({\mr x}\,;{\gr\bar a})\in p}{\IFF}{\theta({\mr x}\,;g\,{\gr\bar a})\in p}\hfill for every $g\in G$.

It should be clear that invariant under the action of $\Aut(\U/A)$ coincides with invariant over $A$ and Lascar invariant over $A$ coincides with invariant under the action of $\Autf(\U/A)$.

Note that $p({\mr x})$ is $G$-invariant exactly when the sets ${\gr\D_{p,\theta}}=\{{\gr b}:\theta({\mr x}\,;{\gr b})\in p\}\subseteq\U^{\gr z}$ are.
Now we would like to discuss invariance using sets $\mrD\subseteq\U^{\mr x}$.

An immediate consequence of the invariance of $\phi(\mrX\,;\grZ)$ is that any $G$-translate of a $\BDelta({\grZ})$-definable set is again $\BDelta({\grZ})$-definable.
In particular for every $\BDelta({\grZ})$-formula $\theta({\mr x}\,;{\gr\bar b})$ and every $g\in G$

\ceq{\hfill g[\theta(\mrX\,;{\gr\bar b})]}{=}{\theta(\mrX\,;g\,{\gr\bar b}).}

Therefore, a type $p({\mr x})\subseteq\BDelta(\grZ)$ is $G$-invariant if

\ceq{\hfill p({\mr x})\proves{\mr x}\in\mrD}{\IFF}{p({\mr x})\proves{\mr x}\in g{\cdot}\mrD}\hfill for every $\BDelta({\grZ})$-definable $\mrD\subseteq\mrX$ and $g\in G$.

% At the first reading the reader may assume that $G=\Aut(\U)$, $\mrX=\U^{\mr x}$, $\grZ=\U^{\gr z}$ and $\Delta=L_{{\mr x}\,;{\gr z}}$, where $|{\gr z}|=\omega$.
% Then $\pmDelta(\grZ)=L_{\mr x}(\U)$ and  $S_\Delta(\grZ)=S_{\mr x}(\U)$. 
% Note that when $G$ acts by automorphisms $\phi(\U^{\mr x};\U^{\gr z})$ is $G$-invariant for all $\phi({\mr x}\,;{\gr z})\in L$.

A set $\mrD\subseteq\mrX$ is \emph{generic\/} under the action of $G$, or \emph{$G$-generic\/} for short, if finitely many $G$-translates of $\mrD$ cover $\mrX$; we say \emph{$n$-$G$-generic\/} if $\le n$ translates suffices.
Dually, we say that $\mrD$ is \emph{persistent\/} under the action of $G$, or \emph{$G$-persistent\/} for short, if the intersection of any finitely many $G$-translates of $\mrD$ is nonempty; we say \emph{$n$-$G$-persistent\/} when the request is limited to $\le n$ translates.
When $\mrX$ and/or $\grZ$ are not clear from the context, we say that these notions are \emph{relative\/} to $\mrX$ and $\grZ$.

The same properties may be attributed to formulas (as these are identified with the set they define).
When these properties are attributed to a type $p({\mr x})$, we understand that they hold for every conjunction of formulas in $p({\mr x})$.

\noindent\llap{\textcolor{red}{\Large\warning}\kern1.5ex}\ignorespaces
The terminology above is non-standard.
In~\cite{CK} the authors write \textit{quasi-non-dividing\/} for \textit{persistent\/} under the action of $\Aut(\U/A)$.
Their terminology has good motivations, but it would be a mouthful if adapted to our context.
In topological dynamics similar notions have been introduced with different terminology: \textit{syndetic\/} corresponds to \textit{generic\/} and \textit{thick\/} corresponds to \textit{persistent.}

\begin{example}
  If $p({\mr x})\subseteq L(\U)$ is finitely satisfiable in $A$ then $p({\mr x})$ is persistent under the action of $G=\Aut(\U/A)$ relative to any $\mrX\supseteq A^{\mr x}$.
  In fact, the same ${\mr a}\in A^{\mr x}$ that satisfies $\phi({\mr x})$ also satisfies every $\Aut(\U/A)$-translate of $\phi({\mr x})$.
\end{example}

Notation: for $\mrD\subseteq\U^{\mr x}$ and $H\subseteq G$ we write \emph{$H\,\mrD$\/} for $\{h\mrD: h\in H\}$.

In this chapter many proofs require some juggling with negations as epitomized by the following fact.

\begin{fact}\label{fact_fip}
  %aaaaa
  The following are equivalent
  \begin{itemize}
    \item[1.] $\mrD$ is not $G$-generic
    \item[2.] $\neg\mrD$ is $G$-persistent.
  \end{itemize}
\end{fact}

\begin{proof}
  Immediate by spelling out the definitions
  \begin{itemize}
    \item[1.] there are no finite $H\subseteq G$ such that $\mrX\ \subseteq\ \cup\, H\,{\mr\D}$.
    \item[2.]  $\0\ \neq\ \mrX\,\cap\,\big(\cap\, H\neg{\mr\D}\big)$ for every finite $H\subseteq G$.\qedhere
  \end{itemize} 
\end{proof}

Define the following type

\ceq{\hfill\emph{$\gamma_G({\mr x})$}}{=}{\{\theta({\mr x})\in \BDelta({\grZ}): \theta({\mr x})\textrm{ is }G\textrm{-generic}\}}

\begin{corollary}\label{corol_q_pers}
  %aaaaa
  Let $p({\mr x})\subseteq\BDelta(\grZ)$ be such that $\gamma_G({\mr x})\cup p({\mr x})$ is finitely satisfiable in $\mrX$.
  Then $p({\mr x})$ is $G$-persistent.
\end{corollary}

\begin{proof}
  Let $\theta({\mr x})$ be a conjunction of formulas in $p({\mr x})$.
  As $\gamma_G({\mr x})$ is finitely satisfiable in $\theta(\mrX)$, it cannot be that $\neg\theta({\mr x})$ is $G$-generic.
  From Fact~\ref{fact_fip}, we obtain that $\theta({\mr x})$ is $G$-persistent.
\end{proof}

The converse implication holds for complete types.

\begin{theorem}\label{thm_generic_invariant}
  %aaaaa
  Let $p({\mr x})\in S_\Delta(\grZ)$.
  Then the following are equivalent
  \begin{itemize}
    \item[1.] $p({\mr x})$ is $G$-invariant and finitely satisfiable in $\mrX$
    \item[2.] $p({\mr x})\proves\gamma_G({\mr x})$
    \item[3.] $p({\mr x})$ is $G$-persistent.
  \end{itemize}
\end{theorem}

\begin{proof}
  \ssf1$\IMP$\ssf2.
  Let $H\subseteq G$ be finite such that $\mrX\subseteq\cup\,H\,\mrD$.
  By completeness and finite satisfiability, $p({\mr x})\proves {\mr x}\in\cup\,H\,\mrD$.
  Again by completeness, $p({\mr x})\proves {\mr x}\in h\,\mrD$ for some $h\in H$.
  Finally, by invariance,  $p({\mr x})\proves{\mr x}\in\mrD$.
  
  \ssf2$\IMP$\ssf3.
  Let $\mrD$ be defined by a conjunction of formulas in $p({\mr x})$.
  If $\mrD$ is not $G$-persistent then, by Fact~\ref{fact_fip}, $\neg\mrD$ is $G$-generic. 
  By \ssf2, $p({\mr x})\proves{\mr x}\notin\mrD$, a contradiction.

  \ssf3$\IMP$\ssf1.
  % uppose $p({\mr x})$ is not $(n+2)$-persistent and let $n$ be minimal.
  % Let $g_1,\dots,g_{n+2}\in G$ and $\theta({\mr x})\,;{\gr\bar b}$ witness this.
  % Then 
  First note that $G$-persistent types are finitely satisfiable in $\mrX$.
  Now, suppose $p({\mr x})$ is not $G$-invariant.
  Then, by completeness, $p({\mr x})\proves\phi({\mr x}\,;{\gr b})\wedge\neg\phi({\mr x}\,;g{\gr b})$ for some $g\in G$.
  Clearly $\phi({\mr x}\,;{\gr b})\wedge\neg\phi({\mr x}\,;g{\gr b})$ is not $2$-$G$-persistent as it is inconsistent with its $g$-translate.
\end{proof}

\begin{corollary}\label{corol_gammaG_invaqriancer}
  The following are equivalent for every $\BDelta(\grZ)$-definable set $\mrD$
  \begin{itemize}
    \item [1.] $\gamma_G({\mr x})\proves{\mr x}\in\mrD$
    \item [2.] $p({\mr x})\proves{\mr x}\in\mrD$ for every $G$-persistent $p({\mr x})\in S_\Delta(\grZ)$.
  \end{itemize}
\end{corollary}

\begin{proof}
  \ssf1$\IMP$\ssf2.
  This is an immediate consequence of Theorem~\ref{thm_generic_invariant}.

  \ssf2$\IMP$\ssf1.
  Suppose $\gamma_G({\mr x})\not\proves{\mr x}\in\mrD$.
  Then there is a type $p({\mr x})\in S_\Delta(\grZ)$ consistent with $\gamma_G({\mr x})\cup\{{\mr x}\notin\mrD\}$.
  By Corollary~\ref{corol_q_pers} $p({\mr x})$ is $G$-persistent.
  Then $\neg\ssf2$.
\end{proof}

The theorem yields a necessary condition for the existence of $G$-invariant global $\BDelta({\grZ})$-types.

\begin{corollary}\label{corol_def_mu}
  %aaaaa
  If there exists a $G$-invariant type $p({\mr x})\in S_\Delta(\grZ)$ finitely satisfiable in $\mrX$ then for every $\BDelta({\grZ})$-definable set $\mrD$
  \begin{itemize}
    \item[1.] $\mrD$ and $\neg\mrD$ are not both $G$-generic
    \item[2.] if $\mrD$ is $G$-generic then it is $G$-persistent
    \item[3.] $\gamma_G({\mr x})$ is finitely satisfiable in $\mrX$.
  \end{itemize}
\end{corollary}

\begin{proof}
  Clearly, \ssf1 and \ssf2 are equivalent by Fact~\ref{fact_fip} and follow from \ssf3.
  Finally, \ssf3 is an immediate consequence of \ssf2 of Theorem~\ref{thm_generic_invariant}.
\end{proof}

The following theorem gives a necessary and sufficient condition for the  existence of global $G$-invariant $\BDelta({\grZ})$-type.
Ideally, we would like to have that every $G$-persistent $\BDelta({\grZ})$-type extends to a global persistent type.
Unfortunately this is not true in general (it requires stronger assumptions, see Section~\ref{tame_landscape}).
A set $\mrD$ is \emph{$G$-wide\/} if every finite cover of $\mrD$ by $\BDelta({\grZ})$-definable sets contains a $G$-persistent set.
In~\cite{CK} a similar property is called \textit{quasi-non-forking.}
Our use of the term \textit{wide\/} is consistent with~\cite{Hr}, though we apply it to a narrow context.
A type is $G$-wide if every conjunction of formulas in the type is $G$-wide.

\begin{theorem}\label{thm_generic_invariant2}
  %aaaaa
  Let $\mrD$ be a $\BDelta(\grZ)$-definable set.
  Then the following are equivalent 
  \begin{itemize}
    \item[1.] $\gamma_G({\mr x})$ is finitely satisfied in $\mrD\cap\mrX$
    \item[2.] there exists a $G$-persistent type $p({\mr x})\in S_\Delta(\grZ)$ containing ${\mr x}\in\mrD$
    \item[3.] $\mrD$ is $G$-wide.
  \end{itemize}
\end{theorem}

\begin{proof}
  \ssf1$\IMP$\ssf2.
  By Corollary~\ref{corol_q_pers}, it suffices to pick any $p({\mr x})\in S_\Delta(\grZ)$ containing $\gamma_G({\mr x})$ and finitely satisfied in $\mrD\cap\mrX$.
  
  \ssf2$\IMP$\ssf1.
  By Theorem~\ref{thm_generic_invariant}.

  \ssf2$\IMP$\ssf3.
  Let ${\mr\C_1},\dots,{\mr\C_n}$ be $\BDelta({\grZ})$-definable sets that cover $\mrD$.
  Pick $p({\mr x})$ as in \ssf2.
  By completeness, $p({\mr x})\proves {\mr x}\in{\mr\C_i}$ for some $i$.
  Then, by Theorem~\ref{thm_generic_invariant}, $\neg{\mr\C_i}$ is not $G$-generic.
  Therefore, by Fact~\ref{fact_fip}, ${\mr\C_i}$ is $G$-persistent.

  \ssf3$\IMP$\ssf2.
  Let $p({\mr x})$ be maximal among the $\BDelta({\grZ})$-types that are finitely satisfiable in $\mrX\cap\mrD$ and are such that $\theta(\U^{\mr x})$ is $G$-wide for every $\theta({\mr x})$ that is conjunction of formulas in $p({\mr x})$.
  We claim that $p({\mr x})$ is a complete $\BDelta({\grZ})$-type.
  Suppose for a contradiction that $\theta({\mr x}),\neg\theta({\mr x})\notin p$.
  By maximality there is some formula $\psi({\mr x})$, a conjunction of formulas in $p({\mr x})$, and some $\BDelta({\grZ})$-definable sets ${\mr\C_1},\dots,{\mr\C_n}$ that cover both $\psi(\U^{\mr x})\cap\theta(\U^{\mr x})$ and $\psi(\U^{\mr x})\smallsetminus\theta(\U^{\mr x})$ and such that no ${\mr\C_i}$ is $G$-persistent.
  As ${\mr\C_1},\dots,{\mr\C_n}$ cover $\psi(\U^{\mr x})$ this is a contradiction.
  It is only left to show that $p({\mr x})$ is $G$-invariant.
  This follows from completeness and Theorem~\ref{thm_generic_invariant}.
\end{proof}

\begin{corollary}\label{corol_intersectionGwide}
  Let $\mrD$ be a $G$-wide $\BDelta(\grZ)$-definable set.
  Then $\mrD\cap g{\cdot}\mrD$ is $G$-wide for every $g\in G$.
\end{corollary}

\begin{proof}
  Let $p({\mr x})\in S_\Delta(\grZ)$ be a $G$-persistent type such that $p({\mr x})\proves{\mr x}\in\mrD$.
  By $G$-invariance $p({\mr x})\proves{\mr x}\in{\gr g}{\cdot}\mrD$. 
\end{proof}

Let $q({\mr x})\subseteq\BDelta(\grZ)$ be $G$-invariant.
We say that $q({\mr x})$ is \emph{$G$-prime\/} if for every $\BDelta(\grZ)$-definable set $\mrD$ and every $g\in G$ if $q({\mr x})\proves{\mr x}\in(\mrD\cup g{\cdot}\mrD)$ then $q({\mr x})\proves{\mr x}\in\mrD$.

\begin{proposition}
  The type $\gamma_G({\mr x})$ is $G$-prime.
\end{proposition}

\begin{proof}
  Suppose $\gamma_G({\mr x})\proves{\mr x}\in(\mrD\cup g{\cdot}\mrD)$.
  Assume that $\gamma_G({\mr x})$ is consistent, otherwise the claim is trivial.
  Let $p({\mr x})\in S_\Delta(\grZ)$ be $G$-persistent.
  We claim that $p({\mr x})\proves{\mr x}\in\mrD$.
  The proposition follows from the claim by Corollary~\ref{corol_gammaG_invaqriancer}.
  By completeness $p({\mr x})\proves{\mr x}\in\mrD$ or $p({\mr x})\proves{\mr x}\in g{\cdot}\mrD$.
  If the latter occurs, $p({\mr x})\proves{\mr x}\in\mrD$ follows from invariance.
\end{proof}


%%%%%%%%%%%%%%%%%%%%%%%%%%
%%%%%%%%%%%%%%%%%%%%%%%%%%
%%%%%%%%%%%%%%%%%%%%%%%%%%
%%%%%%%%%%%%%%%%%%%%%%%%%%
%%%%%%%%%%%%%%%%%%%%%%%%%%
\section{Strong genericity}\label{strong_genericity}

Unfortunately, $G$-genericy is not preserved under intersection.
To obtain closure under intersection, we need to push the concept to a higher level of complexity.

A set $\mrD\subseteq\U^{\mr x}$ is \emph{strongly $G$-generic\/} if for every finite $H\subseteq G$ the set $\cap\,H\,\mrD$ is generic (recall that $H\,\mrD$ stands for $\{h{\cdot}\mrD: h\in H\}$).
Dually, we say that $\mrD$ is \emph{weakly $G$-persistent\/} if for some finite $H\subseteq G$ the set $\cup\,H\,\mrD$ is persistent.
Again, the same properties may be attributed to formulas and types.

\begin{lemma}\label{lem_strongly_generic}
  %aaaaa
  The intersection of two strongly $G$-generic sets is strongly $G$-generic.
\end{lemma}

\begin{proof}
  We may assume that all sets mentioned below are subsets of $\mrX$.
  Let ${\mr\D}$ and ${\mr\C}$ be strongly $G$-generic and let $K\subseteq G$ be an arbitrary finite set.
  It suffices to prove that $\mrB=\cap\, K\,({\mr\C}\cap{\mr\D})$ is $G$-generic. 
  Clearly $\mrB={\mr\C'}\cap{\mr\D'}$, where ${\mr\C'}=\cap\, K\,{\mr\C}$ and ${\mr\D'}=\cap\, K\,{\mr\D}$.
  Note that ${\mr\C'}$ and ${\mr\D'}$ are both strongly $G$-generic.
  In particular $\mrX=\cup\,H\,\mr\D'$ for some finite $H\subseteq G$.
  Now, from
  
  \ceq{\hfill\cup\,H\,\mrB}{=}{\cup\,H\Big[{\mr\C'}\ \cap\ {\mr\D'}\Big]}

  \ceq{\hfill\cup\,H\,\mrB}{\supseteq}{\cup\,H\Big[\big(\cap\, H\,{\mr\C'}\big)\ \cap\ {\mr\D'}\Big]}
  
  \ceq{}{=}{ \big(\cap\, H\,{\mr\C'}\big)\ \cap\ \big(\cup\,H\,{\mr\D'}\big)}
  
  \ceq{}{=}{\cap\, H\,{\mr\C'}}
  
  As ${\mr\C'}$ is strongly $G$-generic, $\cap\, H\,{\mr\C'}$ is $G$-generic.
  Therefore $\cup\,H\,\mrB$ is also $G$-generic.
  The $G$-genericity of $\mrB$ follows.
\end{proof}

Define the following type

\ceq{\hfill\emph{${}^{\rm s}\kern-.2ex\gamma_G({\mr x})$}}{=}{\{\theta({\mr x})\in \BDelta({\grZ}): \theta({\mr x})\textrm{ is strong }G\textrm{-generic}\}}

\begin{corollary}\label{corol_str_gen}
  %aaaaa
  The type ${}^{\rm s}\kern-.2ex\gamma_G({\mr x})$ is finitely satisfiable in $\mrX$, strongly $G$-generic, and $G$-invariant.
\end{corollary}

\begin{proof}
  The strong $G$-genericity is an immediate consequence of Lemma~\ref{lem_strongly_generic}.
  The finite satisfiability is a consequence of $G$-genericity.
  As for invariance, note that any translate of a strongly $G$-generic formula is also strongly $G$-generic.
\end{proof}

\begin{corollary}\label{corol_q_w_pers}
  %aaaaa
  Let $p({\mr x})\subseteq L(\U)$ be such that ${}^{\rm s}\kern-.2ex\gamma({\mr x})\cup p({\mr x})$ is finitely satisfiable in $\mrX$.
  Then $p({\mr x})$ is weakly $G$-persistent.
\end{corollary}

\begin{proof}
  Similar to Corollary~\ref{corol_q_pers}.
  Let $\theta({\mr x})$ be a conjunction of formulas in $p({\mr x})$.
  As ${}^{\rm s}\kern-.2ex\gamma(\mrX)$ is finitely satisfiable in $\theta(\mrX)$, it cannot be that $\neg\theta({\mr x})$ is strongly $G$-generic.
  From Fact~\ref{fact_fip}, we obtain that $\neg\theta({\mr x})$ not being strongly $G$-generic is equivalent to $\theta({\mr x})$  being weakly $G$-persistent.
\end{proof}

%%%%%%%%%%%%%%%%%%%%%%%%%%%%%%%%%%%
%%%%%%%%%%%%%%%%%%%%%%%%%%%%%%%%%%%
%%%%%%%%%%%%%%%%%%%%%%%%%%%%%%%%%%%
%%%%%%%%%%%%%%%%%%%%%%%%%%%%%%%%%%%
%%%%%%%%%%%%%%%%%%%%%%%%%%%%%%%%%%%
\section{The diameter of a Lascar type}\label{newelski}

As an application we prove an interesting theorem of Newelski's on Lascar types.
Let $\Ll({\mr a}/A)$, the set of tuples with the same Lascar strong type as ${\mr a}$ over $A$.
This set is the union of a chain of type-definable sets of the form $\big\{{\mr x}\ :\ d_A({\mr a},{\mr x})\le n\big\}$, where $d_A$ is the distance in the Lascar graph.
In this section we prove that $\Ll({\mr a}/A)$ is type-definable (if and) only this chain is finite.
In other words, only if the connected component of ${\mr a}$ in the Lascar graph has finite diameter.

It is convenient to address the problem in more general terms.
We work under Assumption~\ref{notation_GXphi} and also assume that $G$ acts transitively on $\mrX$ i.e.\@ $G\,{\mr a}=\mrX$ for every ${\mr a}\in\mrX$.
Let $K\subseteq G$ be a set of generators that is
\begin{itemize}
  \item[1.] symmetric i.e.\@ it contains the unit and is closed under inverse
  \item[2.] conjugacy invariant i.e.\@ $g\,Kg^{-1}=K$ for every $g\in G$
\end{itemize}

We define a discrete metric on $\mrX$.
For ${\mr a},{\mr b}\in\mrX$ let $d({\mr a},{\mr b})$ be the minimal $n$ such that ${\mr a}\in K^n{\mr b}$.
This defines a metric which is $G$-invariant by \ssf2.
The \emph{diameter\/} of a set $\mrC\subseteq\mrX$ is the supremum of $d({\mr a},{\mr b})$ for ${\mr a},{\mr b}\in\mrC$.

We are interested in sufficient conditions for $\mrX$ to have finite diameter.
The notions introduced in Section~\ref{strong_genericity} offer some hint.

\begin{proposition}\label{prop_wpers_finite_diameter}
  If $\mrX$ has a weakly persistent subset of finite diameter, then $\mrX$ itself has finite diameter.
\end{proposition}

\begin{proof}
  Let $\mrC\subseteq\mrX$ be a weakly persistent set of diameter $n$.
  Let $H\subseteq G$ be finite such that $\cup\,H\,\mrC$ is persistent.
  We claim that also $\cup\,H\,\mrC$ has finite diameter.
  Let ${\mr a}\in\mrC$ be arbitrary.
  Let $m$ be larger than $d(h{\mr a}, k{\mr a})$ for all $h,k,\in H$.
  Now, let $h{\mr b}$ and $k{\mr c}$, for some $h,k,\in H$ and ${\mr b},{\mr c}\in\mrC$, be two arbitrary elements of $\cup\,H\,\mrC$.
  As $h\mrC$ and $k\mrC$ have the same diameter of $\mrC$, 

  \ceq{\hfill d(h{\mr b},\, k{\mr c})}{\le}{d(h{\mr b},\,h{\mr a})\ +\ d(h{\mr a},\, k{\mr a})\ +\ d(k{\mr a},\,k{\mr c})}

  \ceq{}{\le}{n+m+n.}

  This proves that $\cup\,H\,\mrC$ has finite diameter.
  Therefore, without loss of generality, we may assume that $\mrC$ itself is persistent.
  
  By the transitivity of the action, any two elements of $\mrX$ are of the form $h{\mr a}$, $k{\mr a}$ for some $h,k\in G$ and some ${\mr a}\in\mrC$.
  By persistency, there are ${\mr c}\in\mrC\cap h\mrC$ and ${\mr d}\in\mrC\cap k\mrC$.
  Then 

  \ceq{\hfill d(h{\mr a},\, k{\mr a})}{\le}{d(h{\mr a},\,{\mr c})\ +\ d({\mr c},\, {\mr d})\ +\ d({\mr d},\,k{\mr a})}

  \ceq{}{\le}{n+n+n.}

  Therefore the diameter of $\mrX$ does not exceed $3n$.
\end{proof}

\begin{theorem}\label{thm_newelski}
  Suppose that $\mrX$ and the sets ${\mr\X_n}=K^n{\mr a}$, for some ${\mr a}\in\mrX$, are type-definable.
  Then $\mrX$ has finite diameter.
\end{theorem}

\begin{proof}
  By Proposition~\ref{prop_wpers_finite_diameter}, it suffices to prove that ${\mr\X_n}$ is weakly persistent.
  By Corollary~\ref{corol_q_w_pers} it suffices to show that for some $n$ the type ${}^{\rm s}\kern-.2ex\gamma_G({\mr x})$ is finitely satisfied in ${\mr\X_n}$.
  Suppose not.
  Let $\psi_n({\mr x})\in{}^{\rm s}\kern-.2ex\gamma_G$ be a formula that is not satisfied in $\mr\X_n$.
  Then the type $p({\mr x})=\{\psi_n({\mr x}):n\in\omega\}$ is finitely satisfied in $\mrX$.
  From the type-definablity of $\mrX$ it follows that $p({\mr x})$ has a realization in $\mrX$.
  As this realization belongs to some ${\mr\X_n}$ we contradict the definition of $\psi_n({\mr x})$. 
\end{proof}

% \ceq{\hfill p_n({\mr\X},{\mr\X})}{=}{\{\<{\mr a},{\mr b}\>\in{\mr\X^2}\ :\ {\mr a}\in K^n{\mr b}\}}

% Below we write ${\mr x}\in K^n{\mr y}$ for the type $p_n({\mr x},{\mr y})$ and ${\mr x}\in G\,{\mr y}$ for the disjunction of all these types (n.b.\@ an infinite disjunction of types need not be a type).

\begin{example}\label{ex_newelski}
  Let $K\subseteq\Aut(\U/A)$ be the set of automorphisms that fix a model containing $A$.
  Then the group $G$  generated by $K$ is $\Autf(\U/A)$ and $G\cdot{\mr a}=\mrX$ is $\Ll({\mr a}/A)$.
  Let $\Delta=L_{{\mr x}\,{\gr z}}$ and $\grZ=\U^{\gr z}$.
  Then $d({\mr a},{\mr b})$ coincides with the distance in the Lascar graph.
  The sets $K^n\cdot{\mr a}=\{{\mr x}:d({\mr x},{\mr a})\le n\}$ are type definable.
  Then from Theorem~\ref{thm_newelski} it follows that $\Ll({\mr a}/A)$ is type definable (if and) only if it has a finite diameter.
\end{example} 

%%%%%%%%%%%%%%%%%%%%%%%%%%%%%%%%%%%
%%%%%%%%%%%%%%%%%%%%%%%%%%%%%%%%%%%
%%%%%%%%%%%%%%%%%%%%%%%%%%%%%%%%%%%
%%%%%%%%%%%%%%%%%%%%%%%%%%%%%%%%%%%
%%%%%%%%%%%%%%%%%%%%%%%%%%%%%%%%%%%
\section{A tamer landscape}\label{tame_landscape}

Under suitable assumptions some notion introduced in this chapter coalesce, and we are left with a tamer landscape.
In general, we have the following.

\begin{theorem}\label{thm_coalesce}
  Assume 
  \begin{itemize}
    \item[1.] $G$-persistent $\BDelta({\grZ})$-definable sets are $G$-wide.
  \end{itemize}
  Then the following hold
  \begin{itemize}
    \item[2.] $G$-generic $\BDelta({\grZ})$-definable sets are closed under intersection 
    \item[3.] $G$-generic $\BDelta({\grZ})$-definable sets are strongly $G$-generic
    \item[4.] weakly persistent $\BDelta({\grZ})$-definable sets are $G$-persistent.
  \end{itemize}
\end{theorem}

\begin{proof}
  Clearly \ssf2$\IFF$\ssf3$\IFF$\ssf4.

  \ssf1$\IMP$\ssf2.
  Let $\mrC$ and $\mrD$ be $G$-generic $\BDelta({\grZ})$-definable sets.
  Suppose for a contradiction that $\mrC\cap\mrD$ is not $G$-generic.
  Then $\neg(\mrC\cap\mrD)$ is $G$-persistent.
  By \ssf1 and Theorem~\ref{thm_generic_invariant2} there is a $G$-invariant global $\BDelta({\grZ})$-type $p({\mr x})$ containing ${\mr x}\notin\mrC\cap\mrD$.
  By completeness either $p({\mr x})\proves{\mr x}\notin\mrC$ or $p({\mr x})\proves{\mr x}\notin\mrD$.
  This is a contradiction because by Theorem~\ref{thm_generic_invariant} $p({\mr x})\proves{\mr x}\in\mrC$ and $p({\mr x})\proves{\mr x}\in\mrD$.
  %
  % \ssf4$\IMP$\ssf1.
  % Note that, by \ssf3, the type ${}^{\rm s}\kern-.2ex\gamma_G({\mr x})$ coincides with $\gamma_G({\mr x})$, in particular $\gamma_G({\mr x})$ is finitely satisfied in $\mrX$.
  % Let $\mrD$ be a $G$-persistent $\BDelta({\grZ})$-definable set.
  % We show that $\gamma_G({\mr x})={}^{\rm s}\kern-.2ex\gamma_G({\mr x})$ is finitely satisfiable in $\mrX\cap\mrD$.
  % Then, by \ssf4 and Corollary~\ref{corol_q_w_pers}, any global extension of $\gamma_G({\mr x})\cup\{{\mr x}\in\mrD\}$ witness \ssf2 of Theorems~\ref{thm_generic_invariant2}.
  % Suppose not, then $\gamma_G({\mr x})\proves {\mr x}\notin\mrD$.
  % Therefore $\neg\mrD$ is $G$-generic, contradicting the consistency of ${}^{\rm s}\kern-.2ex\gamma_G({\mr x})$.
\end{proof}

\begin{question}
  Is \ssf1 in Theorem~\ref{thm_coalesce} equivalent to \ssf2-\ssf4?
\end{question}

% \begin{assumption}\label{notation_2}
%   For $G$, $\mrX$, $\grZ$ and $\Delta$ as in Assumption~\ref{notation_GXphi} we also require that the equivalent conditions in Theorem~\ref{thm_coalesce} hold.
% \end{assumption}

\begin{remark}\label{rem_coalesce}
  Assume \ssf2-\ssf4 in Theorem~\ref{thm_coalesce} hold.
  Then the types $\gamma_G({\mr x})$ and ${}^{\rm s}\kern-.2ex\gamma_G({\mr x})$ coincide, and therefore $G$-invariant global types exist.
  It is also worth mentioning that every positive Boolean combination of $G$-generic sets is $G$-generic.
\end{remark}

%%%%%%%%%%%%%%%%%%%%%%%%%%
%%%%%%%%%%%%%%%%%%%%%%%%%%
%%%%%%%%%%%%%%%%%%%%%%%%%%
%%%%%%%%%%%%%%%%%%%%%%%%%%
%%%%%%%%%%%%%%%%%%%%%%%%%%
\section{The action of normal subgroups}\label{normalsubgroups}

Let $H\trianglelefteq G$.
The following is an immediate consequence of normality.

\begin{remark}\label{rem_invariance_normalsubg}
\newlength{\ceqlength}
\settowidth{\ceqlength}{p(x) is H-invariant\ }
\def\medrel#1{\parbox[t]{5ex}{$\displaystyle\hfil #1$}}
\def\ceq#1#2#3{\parbox[t]{\ceqlength}{$\displaystyle #1$}\medrel{#2}{$\displaystyle #3$}}
%
  For every $\mrD\subseteq\U^{\mr x}$ and every $g\in G$
  
  \ceq{\hfill\mrD\textrm{ is }H\textrm{-foo}}{\IFF}{g{\cdot}\mrD\textrm{ is }H\textrm{-foo},}
  
  where \textit{foo\/} can be replaced by \textit{generic,} \textit{invariant,} \textit{persistent,} \textit{wide.}
  In particular, the type $\gamma_H({\mr x})$ is $G$-invariant.
  % It also follows that for any $p({\mr x})\in S_\Delta(\grZ)$
  % \begin{itemize}
  %   \item [3.] \ceq{\hfill p({\mr x})\textrm{ is }H\textrm{-invariant}}{\IFF}{g\cdot p({\mr x})\textrm{ is }H\textrm{-invariant}.}
  % \end{itemize}
\end{remark}

Recall that if $\gamma_H({\mr x})$ is finitely satisfiable in $\mrX$, then $H$-generic sets are $H$-wide, cf.\@ Theorem~\ref{thm_generic_invariant2}.
As it happens, we can slightly strengthen this fact.

\begin{proposition}\label{prop_Ggeneric_Hpersistent}
  Assume that $\gamma_H({\mr x})$ is consistent.
  Let $\mrD$ be a $\BDelta(\grZ)$-definable set.
  Then if $\mrD$ is $G$-generic it is also $H$-wide.
\end{proposition}

\begin{proof}
  Let $p({\mr x})\in S_\Delta(\grZ)$ be consistent with $\gamma_H({\mr x})$.
  As $\mrD$ is $G$-generic, by completeness $p({\mr x})\proves{\mr x}\in g{\cdot}\mrD$ for some $g\in G$.
  Equivalently, $g^{-1}\!\cdot p\proves{\mr x}\in\mrD$.
  As $p({\mr x})$ is $H$-persistent, by Remark~\ref{rem_invariance_normalsubg} $g^{-1}\!\cdot p({\mr x})$ is also $H$-persistent.
  Then the proposition follows from Theorem~\ref{thm_generic_invariant2}.
\end{proof}

%%%%%%%%%%%%%%%%%%%%%%%%%%
%%%%%%%%%%%%%%%%%%%%%%%%%%
%%%%%%%%%%%%%%%%%%%%%%%%%%
%%%%%%%%%%%%%%%%%%%%%%%%%%
%%%%%%%%%%%%%%%%%%%%%%%%%%
\section{Definable groups}\label{definablegroups}

\def\medrel#1{\parbox[t]{5ex}{$\displaystyle\hfil #1$}}
\def\ceq#1#2#3{\parbox[t]{25ex}{$\displaystyle #1$}\medrel{#2}{$\displaystyle #3$}}

In this section we assume that $\grZ$ and $\mrX$ are type-definable over some set of parameters $A$.
Moreover we assume that $\grZ$ is a group that act on $\mrX$.
The group operations and the group action are assumed definable over $A$.
We use the symbol $\,\cdot\,$ for both the group multiplication and the group action.

Let $\Psi\subseteq L_{\mr x}(\U)$ be some small set of formulas.
In this section $\Delta$ contains formulas $\phi({\mr x}\,;{\gr z})$ of the form  $\psi({\gr z^{-1}}\!\cdot{\mr x})$ for $\psi({\mr x})\in\Psi$.
The sets $\phi(\mrX\,;\grZ)$ are $\grZ$-invariant.
We write ${\gr 1}$ for the identity of $\grZ$.
If $\phi(\mrX\,;{\gr 1})\in\BDelta({\gr 1})$ then $\phi(\mrX\,;{\gr g})={\gr g}\cdot\phi(\mrX\,;{\gr 1})$.

The following auxiliary structure is useful.
Let $\UDelta=\big\<\mrX\,;\grZ\big\>$ be a 2-sorted structure whose signature $\LDelta$ contains relation symbols for every formula $\phi({\mr x}\,;{\gr z})\in\Delta$.
As there is little risk of confusion, these relations symbols are also denoted by $\phi({\mr x}\,;{\gr z})$.

To each ${\gr h}\in\grZ$ we associate the $\LDelta$-automorphism $\<{\mr a}\,;{\gr g}\>\mapsto\<{\gr h}{\cdot}{\mr a}\,;{\gr h}{\cdot}{\gr g}\>$.
Therefore $\grZ$ is, up to isomorphism, a subgroup of $G$.
In fact, it is a normal subgroup.
For any ${\gr g}\in\grZ$, the orbit of $\phi(\mrX\,;{\gr g})$ under the action of $\grZ$ is $\{\phi(\mrX\,;{\gr h}) : {\gr h}\in\grZ\}$.
Therefore it coincides with the orbit under the action of $G$.

Now, we consider the action some other normal subgroup \emph{$H$\/} $\trianglelefteq G$.

\begin{proposition}\label{prop_Ggeneric_persistent}
  Let $\mrD$  be a $\BDelta$-definable set.
  Assume that $\gamma_H({\mr x})$ is consistent.
  Then \ssf1$\IMP$\ssf2 holds, where
  \begin{itemize}
    \item [1.] $\mrD$ is $\grZ$-generic
    \item [2.] ${\gr g}{\cdot}\mrD$ is $H$-wide for every ${\gr g}\in\grZ$.
  \end{itemize}
\end{proposition}

Under the assumption of stability and with $H=\Autf(\U^\Delta)$ a stronger claim obtains~--~the consistency of $\gamma_H({\mr x})$ is guaranteed, and also the converse implication holds
%
\begin{proof}
  Let ${\gr g}$ be given.
  If $\mrD$ is $\grZ$-generic, then so is ${\gr g}{\cdot}\mrD$.
  Then ${\gr g}{\cdot}\mrD$ is, a fortiori, $G$-generic.
  Therefore \ssf2 follows from Proposition~\ref{prop_Ggeneric_Hpersistent}.
\end{proof}

We write \emph{$({\gr g})_H$\/} for the $H$-orbit of ${\gr g}$, that is, the set $\{f({\gr g})\ :\ f\in H\}$.

\begin{proposition}\label{prop_wideHcojugate}
  Let $\theta({\mr x}\,;{\gr z_1},\dots,{\gr z_n})$ be a Boolean combination of formulas $\phi_i({\mr x}\,;{\gr z_i})$ for some $\phi_i({\mr x}\,;{\gr z})\in\Delta$.
  Then for every ${\gr h_i}\in({\gr g_i})_H$ the following are equivalent
  \begin{itemize}
    \item [1.] $\theta({\mr x}\,;{\gr g_1},\dots,{\gr g_n})$ is $H$-wide
    \item [2.] $\theta({\mr x}\,;{\gr h_1},\dots,{\gr h_n})$ is $H$-wide.
  \end{itemize}
\end{proposition}

\begin{proof}
  Without loss of generality we can assume that only conjunctions occur in $\theta({\mr x}\,;{\gr g_1},\dots,{\gr g_n})$.
  Let ${\mr\C_i}=\phi_i(\mrX\,;{\gr 1})$.
  Then \ssf1 says that $\mrC={\gr g_1}{\cdot}{\mr\C_1}\cap\dots\cap{\gr g_n}{\cdot}{\mr\C_n}$ is $H$-wide.
  Let $f_i\in H$ be such that ${\gr h_i}=f_i({\gr g_i})$.
  Then, by Corollary~\ref{corol_intersectionGwide} also the intersection of the sets $f_i[\mrC]$ is $H$-wide.
  A fortiori the intersection of the sets $f_i[{\gr g_i}{\cdot}{\mr\C_i}]$ is $H$-wide.
  As  $f_i[{\gr g_i}{\cdot}{\mr\C_i}]={\gr h_i}{\cdot}{\mr\C_i}$, implication \ssf1$\IMP$\ssf2 follows and, by symmetry, this proves the equivalence.
\end{proof}

If $\grA\subseteq\grZ$, we write \emph{$\<\grA\>$} for the subgroup generated by $\grA$.

\begin{proposition}\label{prop_stabilizer1}
  Let $\theta({\mr x}\,;{\gr z_1},\dots,{\gr z_n})$ be a Boolean combination of formulas $\phi_i({\mr x}\,;{\gr z_i})$ for some $\phi_i({\mr x}\,;{\gr z})\in\Delta$.
  Let ${\gr g}\in\grZ$ be arbitrary.
  Assume that $\theta({\mr x}\,;{\gr 1},\dots,{\gr 1})$ is $H$-wide.
  Then $\theta({\mr x}\,;{\gr h_1},\dots,{\gr h_n})$ is $H$-wide for every 
  
  \ceq{\hfill{\gr h_1},\dots,{\gr h_n}}{\in}{\Big\<\bigcup_{{\gr g}\in\grZ}({\gr g})_H^{-1}\!\cdot({\gr g})_H\Big\>.}
\end{proposition}

\begin{proof}
  We proceed by induction on the number of factors of the form ${\gr a^{-1}}\!\cdot{\gr b}$, for some ${\gr a},{\gr b}\in({\gr g})_H$, that occur in ${\gr h_1},\dots,{\gr h_n}$.
  Without loss of generality we can assume that only conjunctions occur in $\theta({\mr x}\,;{\gr 1},\dots,{\gr 1})$.
  Let ${\mr\C_i}=\phi_i(\mrX\,;{\gr 1})$.
  Assume inductively that ${\gr h_1}{\cdot}{\mr\C_1}\cap\dots\cap{\gr h_n}{\cdot}{\mr\C_n}$ is $H$-wide.
  Pick two arbitrary ${\gr a},{\gr b}\in({\gr g})_H$.
  By Remark~\ref{rem_invariance_normalsubg}
  
  \hspace*{7ex}${\gr a}\cdot{\mr\C_1}\ \cap\ {\gr a}\cdot{\gr h_1^{-1}}\!\cdot{\gr h_2}\cdot{\mr\C_2}\ \cap\ \dots\dots\ \cap\ {\gr a}\cdot{\gr h_1^{-1}}\!\cdot{\gr h_n}\cdot{\mr\C_n}$ is $H$-wide.
  
  By Proposition~\ref{prop_wideHcojugate}, we can replace ${\gr a}\cdot{\mr\C_1}$ by ${\gr b}\cdot{\mr\C_1}$.
  Then, again by Remark~\ref{rem_invariance_normalsubg}

  \hspace*{7ex}${\gr h_1}\cdot{\gr a^{-1}}\!\cdot{\gr b}\cdot{\mr\C_1}\ \cap\ {\gr h_2}\cdot{\mr\C_2}\ \cap\ \dots\dots\ \cap\ {\gr h_n}\cdot{\mr\C_n}$ is $H$-wide.
\end{proof}

\begin{question}
  Let $H=\Autf(\UDelta)$.
  Under the assumption of stability the group in Proposition~\ref{prop_stabilizer1} has finite index in $\grZ$ and is contained in the connected component of $\grZ$.
  Does it coincide with it? I.e.\@ is it definable?
\end{question}

\begin{question}
  Let $H=\Autf(\UDelta)$.
  Describe the group in Proposition~\ref{prop_stabilizer1}.
\end{question}

\begin{question}
  Is the group in Proposition~\ref{prop_stabilizer1} $H$-invariant?
\end{question}

%%%%%%%%%%%%%%%%%%%%%%%%%%
%%%%%%%%%%%%%%%%%%%%%%%%%%
%%%%%%%%%%%%%%%%%%%%%%%%%%
%%%%%%%%%%%%%%%%%%%%%%%%%%
%%%%%%%%%%%%%%%%%%%%%%%%%%
\section{Notes and references}

\begin{biblist}[]\normalsize
  \bib{CK}{article}{
   author={Chernikov, Artem},
   author={Kaplan, Itay},
   title={Forking and dividing in ${\rm NTP}_2$ theories},
   journal={J. Symbolic Logic},
   volume={77},
   date={2012},
  %  number={1},
   pages={1--20},
  %  issn={0022-4812},
  %  review={\MR{2951626}},
  %  doi={10.2178/jsl/1327068688},
}

\bib{Hr}{article}{
  label={Hr},
  author={Hrushovski, Ehud},
  title={Stable group theory and approximate subgroups},
  journal={J. Amer. Math. Soc.},
  volume={25},
  date={2012},
  number={1},
  pages={189--243},
  %issn={0894-0347},
  %doi={10.1090/S0894-0347-2011-00708-S},
}
  \end{biblist}

\end{document}
